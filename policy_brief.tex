%%----------------------------------------------------------
%% PolicyBrief.tex
%% v1.1
%% 2022/11/25
%% a template for Policy Brief Homeworks
%% Allame Tabatabaei University, Tehran
%% by Samad Nowruzian  E-mail: nowruzian@gmail.com
%% based on XIE Yuhao template
%%----------------------------------------------------------
\documentclass[12pt]{atu}
\title{ مدیریتِ آلودگی هوای تهران}
\author{صمد نوروزیان}
%\affil{دانشکده حقوق و علوم سیاسی }
\makeatletter
\def\figurename{\if@RTL نمودار\else Diagram\fi}
\makeatother
\makeatother
\begin{document}
\maketitle

\section{چهارچوب و اهمیت مسئله}
آنچه در این خلاصه‌ی سیاستی خواهید خواند، در واقع ارائه‌ی روایتی خلاصه از وضعیتِ چند سال اخیرِ آلودگی هوای تهران، دلایل ایجاد و ماندگاری این آلودگی و تاثیرِ آن بر زندگیِ ما، دلایل ناموفق بودن مجموعه‌ی اقدامات انجام شده تاکنون و در نهایت ارائه‌ی راهکارهایی برای برای برون رفت از این بحران است.


تهران سالهاست که با آلودگی هوا دست به گریبان است و یکی از آلوده‌ترین شهر‌های دنیا به شمار می‌آید.\cite{tehranairworldbank}
نمودار شماره‌ی
\ref{fig:10yearsdays}
وضعیت ده سال اخیر هوای تهران را نمایش می‌دهد. همان‌طور که مشخص است، هوای تهران در ده سالِ اخیر در بیش از یک چهارم روزهای سال در وضعیتِ ناسالم\RTLfootnote{منظور از هوای ناسالم، مجموع روزهای ناسالم برای گروه حساس، ناسالم، بسیار ناسالم و خطرناک طبق معیارهای سازمان کنترلِ کیفیتِ هوای تهران است.}
قرار داشته است.
\vspace{-0.3cm}
\begin{figure}[!ht]
	\centering
	\includegraphics[width=\columnwidth]{10yearsdays.eps}
	\caption{وضعیت هوای تهران در سال‌های ۱۳۹۰ تا ۱۴۰۰} \label{fig:10yearsdays}
\end{figure}
\vspace{-0.3cm}
میزان آلاینده‌ی ذراتِ معلق با قطرِ کمتر از $2.5$ میکرون 
{\scriptsize (\lr{$PM_{2.5}$}) }
به عنوان اصلی‌ترین آلاینده هوای تهران، درسال ۱۴۰۰ در اغلب روزهای سال بالاتر از استاندارد جهانی و در روزهای زیادی ۳ تا ۵ برابر استاندارد بوده است 
\cite{tehranair1400}

بر اساس محاسبات بانکِ جهانی و مراکزِ تحقیقاتیِ داخلی خساراتِ اقتصادی آلودگی هوای تهران در سال ۲۰۱۸ مبلغ $2.2$  تا $2.6$  میلیارد دلار برآورد شده است.\cite{Faridi2022-rz} 
همچنین در بحث سلامت، تحقیقات نشان داده است قرار گرفتن در معرضِ ذراتِ معلق با قطرِ کمتر از $2.5$ میکرون با مرگ‌و‌میرِ ناشی از بیماری ایسکمیک قلب، سکتهٔ مغزی، بیماری مزمنِ انسدادی ریه، سرطانِ ریه و همچنین عفونت‌های حادِ دستگاهِ تنفسیِ تحتانی در کودکان مرتبط است. همچنین آلودگی هوا به عنوان اصلی‌ترین عاملِ خطرِ محیطی در بروز و پیشرفت برخی از بیماری‌ها مانند: آسم، هیپرتروفی بطن، آلزایمر و پارکینسون، عوارضِ روانی، اُوتیسم، رتینوپاتی، اختلالِ رشدِ جنین و وزنِ کمِ هنگامِ تولد شناخته می‌شود.\cite{Ghorani-Azam2016-xh} 
\cite{salamat1395}
نتایجِ مطالعات همچنین تعدادِ مرگ‌ومیرِ زودرسِ منتسب به آلودگی هوای تهران را سالانه بیش از ۴ هزار نفر برآورد می‌کنند.
\cite{tehranairworldbank}

شرایط توپوگرافیکِ تهران و وجود کوه‌ها در شرق و شمال تهران، همانند دیواری در مقابل خروج آلودگی عمل می‌کنند. وزش بادهای غربی و جنوب غربی و استقرار صنایع در غرب و جنوب تهران باعث انتقال آلودگی به سطح شهر تهران و تشدید آلودگی می‌شود. همچنین بالاتر بودن دمای هوای مرکز شهر تهران نسبت به حومه‌‌ی آن سبب ایجاد جزیره حرارتی در مرکز و باعث وزش جریان باد از اطراف به سمت شهر و هدایت آلاینده‌های حومه به داخل شهر می‌شود. با این وجود مهمترین عامل اقلیمی موثر در آلودگی هوای تهران وارونگی‌های دمایی است که عامل اصلی ماندگاری آلاینده‌ها بخصوص در روزهای سردِ سال است.\cite{safavi}\cite{lashkari}
بلندمرتبه‌سازی در بخش شمالی شهر مانع از وزش بادهای کوه به دشت شده و همچنین توسعه بلندمرتبه‌سازی در غرب تهران، وزیدن بادهای غربی بعنوان باد غالب تهران را دچار اخلال کرده است.\cite{salighe}

جمعیتِ تهران بر اساس آخرین سرشماری، بیش از هشت و نیم میلیون نفر است که این جمعیت در طول روز تا دوازده میلیون نفر افزایش می‌یابد.
\cite{tehranairworldbank}
این میزان جمعیت، باعث ایجاد  $19.8$ میلیون سفر شهری در روز می‌شود که سیستمِ حمل‌و‌نقلِ عمومی تنها توانِ پوششِ ۴۴ درصد\RTLfootnote{برخی اعضای شورای شهر تهران، میزان سفر شهری  را تا ۲۳ میلیون سفر و سهم حمل و نقل عمومی را حداکثر تا ۱۳ درصد بیان کرده است}
  از این میزان سفر را دارد.
\cite{amarname1400}

منابعِ تولیدِ آلاینده‌های هوای شهر تهران به دو گروه منابعِ متحرک: خودروهای سواری شخصی، تاکسی‌ها، موتورسیکلت، مینی‌بوس‌ها، اتوبوس‌های سرویس و شرکت واحد، خودروهای باری سبک و سنگین و منابعِ ساکن: صنایع، خانگی، تجاری، تبدیل انرژی( نیروگاه و پالایشگاه)، پایانه‌های مسافربری و جایگاه‌های عرضه سوخت، تقسیم می‌شوند. در نمودار شماره‌ی 
\ref{fig:alayande}
می‌توانید سهم آلایندگی هر یک از منابع متحرک و ساکن به تفکیک نوع آلاینده را مشاهده کنید.
\cite{hemayat1400}

\vspace{-0.4cm}
\begin{figure}[!ht]
	\centering
	\includegraphics[width=\columnwidth]{alayande.eps}
	\caption{منبع آلاینده‌های هوای تهران به تفکیک متحرک یا ساکن بودن} \label{fig:alayande}
\end{figure}
\vspace{-0.4cm}

تعداد $4.2$ میلیون وسیله نقلیه در تهران در تردد است که ٪۷۲ آن را سواری شخصی، ٪18 موتورسیکلت و ٪10 دیگر را تاکسی، اتوبوس، مینی‌بوس، وانت و کامیون تشکیل می‌دهد. 
موتورسیکلت با تولید میزان زیادی از مونوکسیدِ کربن و ذراتِ معلق یکی از نگرانی‌های اصلی کیفیت هوای تهران تبدیل شده است. خودروهای متوسط و سنگین (مینی‌بوس، اتوبوس و کامیون) تنها $2.4$ درصد از ناوگان حمل‌و‌نقلِ تهران را تشکیل می‌دهند، اما به ترتیب بیش از 41، 64 و 85 درصد از آلاینده‌های
{\scriptsize \lr{$NO_{x}$}}،
{\scriptsize \lr{$SO_{x}$}}  و
{\scriptsize \lr{$PM_{2.5}$}}  
را تولید می‌کنند.
نمودار شماره‌ی 
\ref{fig:PM25}
به خوبی نشان می‌دهد که کامیون‌ها، اتوبوس‌های شهرداری و اتوبوس‌های بخشِ خصوصی سهمی ۷۸ درصدی در تولید آلاینده 
 {\scriptsize \lr{$PM_{2.5}$}}
را بر عهده دارند. سهم ۱۲ درصدی موتورسیکلت نیز قابل تأمل است. 
\cite{Shahbazi_2016}
\vspace{-0.4cm}
\begin{figure}[!ht]
	\centering
	\includegraphics[width=\columnwidth]{PM25.eps}
	\caption{درصد سهم وسایل نقلیه در انتشار {\scriptsize \lr{$PM_{2.5}$}}} \label{fig:PM25}
\end{figure}
\vspace{-0.4cm}

وسایل نقلیه سنگین وضعیت خوبی از نظر سن ندارند. نمودار شماره‌ی
\ref{fig:age}
وضعیت این توزیع سنی را نشان می‌دهد. نمودار نشان مید‌هد ۲۱ درصد از خودروهای سواری بیش از ۱۰ سال سن دارند، همچنین مینی‌بوس‌ها، اتوبوس‌های بخش خصوصی، اتوبوس‌های شهرداری و کامیون‌ها به ترتیب با ۷۲، ۵۴، ۴۰ و  ۳۸ درصد، بیشترین سهم از خودروهای بالاتر از ۱۰ سال را تشکیل می‌دهند. درصد خودروهای کاربراتوری در خودروهای سواری، تاکسی و وانت به ترتیب$9.37$، $4.76$ و $22.29$ درصد است. بیش از 90 درصد موتورسیکلت‌ها تک‌سیلندر هستند.
\cite{Shahbazi_2016}

\vspace{-0.3cm}
\begin{figure}[!ht]
	\centering
	\includegraphics[width=\columnwidth]{ages.eps}
	\caption{وضعیت سن وسایل نقلیه شهر تهران} \label{fig:age}
\end{figure}
\vspace{-0.4cm}

 با وجود اینکه سن اسقاط(فرسودگی) خودروها 15 سال
 \RTLfootnote{در آخرین نسخه از قانون عبارت «سن فرسودگی » به عبارت «سن مرز فرسودگی» تغییر پیدا کرده است.}
  تعیین شده ولی به دلایل متعدد حجم انبوهی از ماشین‌های با عمر، آلایندگی و مصرف سوخت بالا در حال تردد هستند. روش انجام معاینه‌ی فنی در ایران سختگیرانه نیست و نه تنها دوره‌ی معافیت از آن طولانی مدت است بلکه بر صحت عملکرد آن ابهامات زیادی وجود دارد. در مواردی حتی بر وجود قطعه‌ی کاتالیست - به عنوان قطعه‌ای که بخش زیادی از آلاینده‌ها، به جز ذرات معلق را قبل از انتشار در هوا حذف می‌کند - هیچ گونه نظارتی نمی‌شود.
\cite{tahami}
در حال حاضر بیش از ۱۲ میلیون لیتر بنزین در تهران سوازنده می‌شود. مصرف بالای خودروهای داخلی در کنار کیفیت پایین سوخت اعم از بنزین و گازوئیل از نظر اکتان پایین و بنزن و گوگرد بالا سهم مهمی در افزایش آلودگی هوای تهران دارند.
 \cite{fuelq1394}
\cite{rahimi2020}
با وجود اینکه استاندارد یورو ۴ در ایران از سال ۱۳۹۳ معرفی شده است، هنوز ناوگان وسایلِ نقلیه‌ی سنگین در تهران تا حد زیادی با استانداردهای یورو ۳ یا کمتر مطابقت دارد و در واقع، اکثر وسایلِ نقلیه‌ی سنگین فقط استانداردِ یورو ۱ را برآورده می‌کنند.
\cite{tehranairworldbank}
\cite{amarnaft1397}
  
\section{نقدِ گزینه‌های سیاستی}
به نظر می‌رسد ایران در بحث قوانینِ کیفیتِ هوا به میزان کافی قانون دارد ولی مثل بسیاری از موارد دیگر، مشکل اجرای قانون است. گزارش دیوان محاسبات کشور نشان می دهد که از بین ۲۲۰ حکم قانونی تنها ۲۰ مورد به صورت کامل انجام شده اند، ۱۴۰ حکم ناقص انجام یافته، ۶۰ حکم نیز فاقد عملکرد بوده‌اند. 
\cite{divan1400}
قانون هوای پاک از سال ۱۳۹۶ جایگزین قانون نحوه جلوگیری از آلودگی هوا شد. تکالیف اصلی این قانون شامل: اسقاط وسایل نقلیه فرسوده، استانداردسازی سوخت مصرفی وسایل نقلیه، بهینه‌سازی مصرف سوخت خانگی و صنایع، توسعه و نوسازی ناوگان حمل و نقل عمومی، تامین مالی و ارزیابی عملکرد مبتنی بر بودجه، درصد و تجهیز نیروگاه‌ها به سیستم‌های کنترلی، توسعه انرژی‌های تجدیدپذیر و استانداردسازی سوخت مصرفی نیروگاه‌ها است. 
\subsection{اسقاط وسایل نقلیه فرسوده}
علی‌رغم وجود قوانین الزام‌آور متعدد در خصوص نحوه از رده خارج کردن خودروهای فرسوده در کشور در سال ۱۳۹۸، تنها ۷۷۰۰ دستگاه خودرو (سبک و سنگین)، در سال ۱۳۹۹، ۱۳ هزار دستگاه و در سال ۱۴۰۰، حدود ۳۵ هزار دستگاه خودرو اسقاط شده است. 
لازم به ذکر است در سال ۱۳۹۷ دولت سن فرسودگی را افزایش داد و تعداد ناوگان فرسوده از لحاظ آماری به نصف کاهش یافت. مصوبات هیئت دولت و مجلس برای الزام خودروسازان به ارائه گواهی اسقاط به میزان ۳۰ درصد خودروهای تولیدی  برای کسب مجوز شماره‌گذاری تاکنون اجرایی نشده است. این میزان در آخرین اصلاحیه به ۲۵ درصد کاهش یافته و در نهایت با فشارِ وزارت صمت و خودروسازان الزام ارائه گواهی اسقاط برای شماره گذاری برداشته و مقرر شد به میزان $1.5$ درصد قیمت خودرو  توسط خودروسازان  به خزانه واریز  و این پرداختی جایگزین گواهی استقاط خودرو شود. در حال حاضر راه‌حل جایگزین دولت بجای اسقاطِ خودروها، افزایش تعداد دفعات معاینه‌ی فنی خودروهای فرسوده است. بررسی‌ها و نظارت‌های مختلف نشان داده که بخش قابل توجهی از گواهی‌های معاینه‌ فنی صادر شده عمدی یا سهوی دچار اشکال بوده و خودروی آلوده به نام خودروی سالم در حال تردد باشد.

مشکلات حقوقی در تعیین سن فرسودگی برای خودروها و  رأی دیوان عدالت اداری به خلاف شرع بودن تعیین سن فرسودگی و ابطال آئین‌نامه ماده 8 قانون هوای پاک و همچنین عدم توجه به مصرف بالای خودروهای فرسوده نیز از نقص‌های قانون اسقاط خودروهاست. اگر خودروهای فرسوده بتوانند سختگیرانه‌ترین استاندارد را  پشت سر بگذارد، کماکان مصرف سوخت بالایی دارند. از آنجا که یک خودروی فرسوده نزدیک به ۲ برابر یک خودروی نو بنزین مصرف می‌کند، جدا از اتلاف منابع، بر آلودگی هوا خواهند افزود.

افزایش تورم و کاهش ارزش پول ملی و در نتیجه افزایش قیمت خودروها ازجمله خودروهای فرسوده باعث عدم تمایل برای از رده خارج کردن خودروهای فرسوده به قیمت ناچیز دربرابر قیمت بازاری آن است. بخصوص وقتی که این خودروهای فرسوده اغلب مربوط به دهک های پایین بوده و در بسیاری موارد وسیله امرار معاش ‌آن‌هاست. عدم وجود مشوق‌های مالی و تسهیلات بانکی در این خصوص،  بر این بحران افزوده است. 
\cite{divan1400}
\cite{nosazi}
\cite{divan1399}
\cite{majles1385}
\cite{majles1400}
\subsection{توسعه و نوسازی ناوگان حمل و نقل عمومی}
وضعیت حمل و نقل عمومی در تهران برای جابجایی روزانه نزدیک به ۲۰ میلیون سفر شهری مناسب نیست و بخش زیادی از آن نیز فرسوده است.اولویت‌های غلط تخصیص بودجه شهر طی سال‌ها از یک سو و عدم همکاری دولت در تامین مالی بودجه تهران به رغم تکالیف متعدد قانونی از سوی دیگر ،موجب کمبود ۱۵۰۰ واگنی  متروی تهران و   روندِ کند و عقب‌ماندگی از برنامه زمانی احداث و توسعه آن شده است. 
\cite{shahrdotir}
از آنجا که بخش زیادی از آلودگی هوای تهران ناشی از فعالیت اتوبوس و مینی‌بوس‌های فرسوده است، عدم توجه به نحوه جایگزنی وسایل نقلیه و عدم تامین منابع مالی مورد نیازبه رغم تکلیف چندباره از جمله دلایل ناکامی در این بخش است.
\subsection{نوسازی و بهبود وسایل نقلیه}
در حال حاضر استاندارد اصلی برای تولید خودروهای بنزنی یورو 5، خودروهای دیزلی یورو۴ به همراه نصب فیلتر ذرات و موتورسیکلت یورو ۴ است. براساس قانون می‌بایست از ابتدای ۱۴۰۱ تمامی خودروهای تولیدی با استاندارد یورو ۶ و موتورسیلکت‌ها با یورو ۵ تولید می‌شدند که اجرای این استانداردها با فشار وزارت صمت و خودروسازان  دچار تعویق‌های یکساله شده است.
در حال حاضرتولید سوخت با استانداردهای یورو ۴ و ۵ هر چند با وجود کسری‌ قابل توجه در حال انجام است، با این وجود بسیاری از خودروهای در حال تردد تکنولوژی احتراق آن‌ها همچنان یورو ۲ و پایین‌تر است که نیاز به کنترل مداوم و نصب  و تعویض کاتالیست دارد. عدم توجه به لزوم انطباق کیفیت سوخت توزیع شده و تکنولوژی وسیله نقلیه برای داشتن کمترین میزان آلودگی و نیز کمترین آسیب به خودرو از جمله مواردی است که نادیده گرفته شده است.
همچنین با وجود تردد شمار زیادِ موتورسیکلت های فرسوده در تهران، تاکنون هیچگونه دستورالعمل و شیوه نامه‌ای درخصوص نحوه اجرای آزمون‌های معاینه فنی موتورسیکلت به مراکز معاینه فنی ابلاغ نشده و امکان ثبت نتایج معاینه فنی موتورسیکلت‌ها به صورت برخط و سیستماتیک در سامانه مربوطه مقدور نیست.
با وجود اینکه قیمت کاتالیست مانع از تعویض داوطلبانه آن توسط دارندگان وسیله نقلیه شده، شواهد نشان می‌دهد که ورود دولت به سرمایه‌گذاری و همکاری در تامین کاتالیست‌ها با توجه به جنبه‌های اقتصادی رفع آلودگی، توجیه پذیر است که تاکنون اقدامی در این خصوص صورت نگرفته است.
در بسیاری از تکالیف قانون هوای پاک مثل تغییر ناوگان پیک‌موتوری و موتورسیکلت‌های دراختیار دستگاه‌های دولتی به موتورسیکلت برقی هیچ اقدامی صورت نگرفته است.
\cite{majles1400}
\cite{majles1400b}
\subsection{استانداردسازی سوخت وسایل نقلیه، صنایع و نیروگاهها}
ماده ۲ قانون هوای پاک، وزارت نفت را مکلف به تولید و توزیع بنزین با استاندارد یورو ۴ می‌کند. متاسفانه آمار دقیقی از نوع و کیفیت بنزین تولیدی منتشر نشده است ولی میزان بنزین و گازوئیل تولیدی یورو ۴ در آمار سال  ۱۳۹۹ کمتر از ۴۰ درصد بنزین و گازوئیل تولیدی کشور و بسیار پایینتر از میزان هدف‌گذاری شده است. طبق قانون، سازمان استاندارد متولی تعیین استاندارد سوخت است ولی بعد از ابلاغ قانون هوای پاک هیچ استاندارد جدیدی اعلام نشده و وزارت نفت کماکان براساس استاندارهای قدیمی‌تر تولیدِ محصول می‌کند.
در حال حاضر میزان بنزین مصرفی کشور بیش از میزان تولید و واردات است و در چنین حالتی شاهد ورود بنزین پتروشیمی‌ها به چرخه توزیع هستیم.از آنجاکه در حالت نرمال تولید بنزین در پالایشگاه انجام می‌گیرد، بنزین تولیدی پتروشیمی غیر استاندارد و حاوی میزان بَنزن بیش از اندازه استاندارد است. شواهد بسیاری بر مشکل‌دار بودن این نوع بنزین تولیدی وجود دارد. متاسفانه اطلاعاتی از میزان بنزین تولید پتروشیمی‌ها و نحوه توزیع آن‌ها وجود ندارد.
استفاده از گازوئیل بعنوان سوخت دوم در زمان افت فشار شبکه گاز کشور، باعث افزایش  شاخص {\scriptsize \lr{$SO_{x}$}} در هوای تهران در آذر و دیماه شده است که نشان از نامرغوب بودن سوخت مصرفی و یا احتراق ناقص دارد. همچنین این افزایش غیر طبیعی غلطت دی اکسید گوگرد بعنوان شاهدی برای سوزاندن مازوت تلقی می‌شود.
\subsection{اجرای طرح منطقه‌ی کم انتشار (\lr{LEZ})}
یکی از روش‌های مفید برای کاهش آلودگی هوا، اجرای طرح منطقه‌ی کم انتشار (\lr{Less Emission Zone})  است که در شهر تهران در حال اجراست.اساس این طرح بر کاهش تردد وسایل نقلیه‌ی آلاینده و نیز تردد محدود با پرداخت هزینه‌های آلودگی هوا بر اساس سن و تکنولوژ‌ی انتشار وسیله‌ی نقلیه است. با این وجود هنوز سازوکار ایجاد محدودیت برای موتورسیکلت‌ها به عنوان منبعِ مهم آلودگی در مرکز شهر اتخاذ نشده است. ترددِ بدون هیچ محدودیتِ حجم انبوهی از موتورسیلکتِ فرسوده در این محدوده، تلاش‌های صورت گرفته برای کاهش تردد وسایل نقلیه آلاینده  در ناحیه  \lr{LEZ} را کم اثر می‌کند.
\subsection{مدیریتِ جزیره‌ای آلودگی هوای تهران}
آنچه در حالِ حاضر بعنوان مجموعه اقدامات دستگاه‌های اجرایی مختلف برای مدیریتِ آلودگی هوای تهران شناخته می‌شود، بیشتر مجموعه‌ای فعالیت‌ِ جزیره‌ای، بدون هماهنگی کافی و در غیاب یک مدیریتِ کلان است. نتیجه آنکه بدون اولویت‌بندی و بدون طی یک روند پایدار، اقدامات اجرایی انجام می‌گیرد که در نهایت نتیجه‌ای کافی، ملموس و کمک کننده به کاهش آلودگی هوای تهران ندارد.
\section{توصیه‌های سیاستی}
به نظر می‌آید با وجود تلاش‌های زیادی که در بحث مدیریت آلودگی هوای تهران صورت گرفته است، به دلایل مختلفی که مورد بحث واقع شد، نتوانسته‌ایم به اهداف مشخص شده دست‌ پیدا کنیم. با وجود گذشتن ۵ سال از تصویب قانون هوای پاک، شهر تهران هنوز آلوده است و در لحظه تدوین این متن در آذر ۱۴۰۱، در یک دوره‌ی طولانی مدتِ هوای ناسالم و تعطیلی مدارس و آشفتگی بسر می‌برد. در بررسی عواملِ عدمِ موفقیتِ سیاست‌های بکارگرفته شده -جدای از عوامل اشاره شده در بخش آسیب‌شناسی سیاست‌ها- به چند دلیل عمده برمی‌خوریم. یک: عدم اولویت‌بندی اقدامات بر اساس بررسی‌های دقیق و وجود ابهامات و اشکالات در بسته‌ سیاستی در حال اجرا، دو:  تعدد قوانین در عین ناهماهنگی و نبود مدیریت پروژه‌ی کاهش آلودگی هوا، سوم: وضعیت مالی و اعتباری کشور و عدم تامین و تخصیص بودجه مورد نیاز.
در این میان ضرورت بازنویسی و اولویت‌بندی راهکارهای کاهش آلودگی هوا بر اساس توجه به  منابع ایجاد آلودگی، بازدهی و تاثیرگذاری راهکار و هزینه و زمان اجرای آن بیش از پیش احساس می‌شود. همچنین توجه و اجرای برنامه‌های کوتاه  مدت با تاثیر سریع در ادامه‌ی برنامه‌های طولانی مدت با تاثیر پایدارتر،ضروت دارد.
اولین اولویت برای سیاستگذاری و ارائه راهکار، کاهش آلایندگی منابع متحرک است.  بر این اساس پیشنهاد می‌شود سیاست‌های زیر در بخش‌های مشخص شده در دستور کار قرار گیرد:
\subsection{بازنگری در ساختار اجرایی}
وزارت کشور با همکاری سازمان برنامه‌و‌بودجه، شهرداری تهران و سازمان محیط زیست، مجموعه اقدامات اجرایی برای کاهش آلودگی هوای تهران را تحت یک مدیریت کلان با ساختار سازمانی چابک و حرفه‌ای با حدودِ اختیارات، امکانات و بودجه‌ی مشخص و در قالبِ یک پروژه‌ی بلند مدت تعریف کند. 
لازم است وظایفِ دستگاه‌های اجرایی مرتبط به بحث، بازنگری شده و تمامی فعالیت‌ها و تکالیف قانونی دستگاه‌های مختلف در عین پیروی از سازوکار درونی سازمان مجری، تحت هماهنگی مدیریت کلانِ پروژه‌ی کاهش آلودگی هوای تهران اجرا و ارزیابی گردد.
\subsection{توسعه‌ی دولتِ الکترونیک}
بخشی از‌ سفرهای شهری، مراجعه به دستگاه‌های اجرایی عمومی، دولتی، قضایی و انتظامی است. بنابراین کاهش سالانه ۲۰ درصد خدمات حضوری و تبدیل به خدمات آنلاین از طریق: توسعه‌ی شیوه‌های احرازِ هویتِ دیجیتال، توسعه‌ی پنجره‌های واحدِ ارائه‌ی خدمت به صورت آنلاین، افزایشِ تعداد و جانمایی درست دفاتر پیشخوان دولت- در مناطق مبدا سفر به مرکز شهر.
\subsection{مدیریتِ کیفیتِ سوختِ نیروگاه‌ها و صنایع}
وزارت نفت با هماهنگی وزارت‌های نیرو و صمت، به شیوه‌ای برنامه‌ریزی نماید از که از سال ۱۴۰۳ تمامی نیروگاه‌ها و صنایع در شعاع ۲۰۰ کیلومتری شهرِ تهران به هیچ وجه از مازوت و سایر سوخت‌های غیراستاندارد استفاده نکنند. از سال ۱۴۰۳ کیفیت سوختِ مصرفی نیز می‌بایست مطابق با استاندارد ملی باشد.
\subsection{تولید و وارداتِ خودرو و موتورسیکلت برقی}
\begin{enumerate}
		\item
 الزام واردکنندگان و  سازندگان خودرو و موتورسلیکت به رعایت  استانداردهای آلایندگی  اتحادیه اروپا و الزام برای انجام آزمون‌های آلایندگی واقعی برای مجوز شماره‌گذاری تمامی خودروهای تولیدی و وارداتی و اعمال نسخه‌های جدید این استاندارها حداکثر با فاصله ۲ ساله. وزارت نفت برنامه‌ریزی لازم برای ارتقای کیفیتِ سوختِ تولیدی را به گونه‌ای انجام دهد که برای خودروهای تولیدی و وارداتی نو، سوختِ استانداردِ متناسب با تکنولوژی ‌آن‌ها در دسترس باشد.
 \item 
برنامه‌ریزی برای تولید خودرو و موتورسیکلتِ هیبرید و  برقی به میزان حداقل ۱۰ درصد کلِ تولیدِ خودروسازان و ۵۰ درصد تولیدِ سازندگان موتورسیکلت تا ۱۴۰۵. سازمان محیط زیست با همکاری سازمان استاندارد، استانداردهای مربوط به خودروهای برقی را متناسب با وضعیت شبکه‌ی برقی کشور تدوین کند.
\item 
راه‌اندازی حداقل ۱۰۰ ایستگاه شارژِ خودرو و موتورسیکلتِ برقی تا سال ۱۴۰۳ با مشارکتِ دولت و سازندگان خودرو و موتورسیکلت در نقاطِ مختلف تهران و تشویق بخشِ خصوصی برای سرمایه‌گذاری در این حوزه.
\end{enumerate}

\subsection{توسعه و نوسازی حمل‌و‌نقلِ عمومی و باری}
\begin{enumerate}
	\item 
	الزامِ کامل تمامی خودروهای حمل‌ونقلِ عمومی و کامیون‌ها به نصب فیلتر جاذب دوده،حداکثر تا سال ۱۴۰۵ به میزان سالانه ۲۵ درصد و جلوگیری از تردد خودروهای سنگین فاقدِ فیلترِ جاذبِ دوده در محدود
\lr{LEZ}
در هر ساعتی از شبانه‌روز.
\item 
ساز-و-کارِ نظارتی به گونه‌ای تعریف شود که از تردد موتورسیکلت‌های فرسوده به
\lr{LEZ}
 کاملاً جلوگیری شود. ورود برای موتورسیکلت‌های انژکتوری با پرداخت هزینه‌‌ی آلایندگی مجاز است و ورود برای موتورسیکلت‌های برقی رایگان است.
\end{enumerate}

\subsection{توسعه و بهبود سنجشِ آلایندگی}
\begin{enumerate}
\item 
شهرداری تهران موظف است نسبت به استقرار سامانه‌های سنجشِ آلایندگیِ از راه‌دور در نقاط مختلف شهر به میزان سالانه ۲ درصدِ معابرِ اصلی شهر اقدام نمائید. بودجه‌ی موردِ نیاز از محلِ عوارضِ ورودی به
\lr{LEZ} 
 تامین می‌شود.
 \item
 شهرداری تهران موظف است نسبت به بررسی مجددِ محلِ استقرار دستگاه‌های ثابت سنجشِ آلودگی هوا برای جانمایی مناسب‌تر آن‌ها اقدام نماید. دستگاه‌های اجرایی که دارای سامانه سنجشِ آلودگی هستند می‌بایست تمامی اطلاعاتِ اندازه‌گیری شده‌ی خود را به صورتِ برخط و لحظه‌ای به سازمان کنترل آلودگی هوای تهران تحویل دهند.
  \item 
 وزارتِ کشور با همکاری سازمان‌های استاندارد و محیط زیست نسبت به فعال‌سازی طرح معاینه‌ی فنی موتورسیکلت‌ها در تهران و اتصال به شبکه سیمفا اقدام نماید.
 \end{enumerate}

\subsection{اسقاط و نوسازی وسایلِ نقلیه‌ی فرسوده}
\begin{enumerate}
	\item 
	جایگزینی تمامی موتورسیکلت‌های کاربراتوری با کاربری حملِ بار و کالای تمامی دستگاه‌های دولتی و عمومی با موتورسیکلت برقی تا سال ۱۴۰۳
	\item 
	الزام کلیه دستگاه‌های دولتی و عمومی به جایگزینی خودروهای فرسوده‌ی موردِ استفاده به میزانِ سالانه ۱۰ درصد و نصب کاتالیست بر روی خودروهای فرسوده به میزانِ سالانه ۲۵ درصد.
	\item 
	به جهت تاکید بر جایگزینی خودروهای فرسوده با نوعِ برقی و یا هیبریدی، جایگزینی خودروهای فرسوده‌ی دولتی می‌بایست با اولویتِ جایگزینی با خودروهای برقی و یا هیبریدی تولیدِ داخل باشد.
\end{enumerate}

%\appendices

\bibliographystyle{unsrt-fa}
\bibliography{citations}

\end{document}
